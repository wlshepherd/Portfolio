% Options for packages loaded elsewhere
\PassOptionsToPackage{unicode}{hyperref}
\PassOptionsToPackage{hyphens}{url}
%
\documentclass[
]{article}
\usepackage{amsmath,amssymb}
\usepackage{iftex}
\ifPDFTeX
  \usepackage[T1]{fontenc}
  \usepackage[utf8]{inputenc}
  \usepackage{textcomp} % provide euro and other symbols
\else % if luatex or xetex
  \usepackage{unicode-math} % this also loads fontspec
  \defaultfontfeatures{Scale=MatchLowercase}
  \defaultfontfeatures[\rmfamily]{Ligatures=TeX,Scale=1}
\fi
\usepackage{lmodern}
\ifPDFTeX\else
  % xetex/luatex font selection
\fi
% Use upquote if available, for straight quotes in verbatim environments
\IfFileExists{upquote.sty}{\usepackage{upquote}}{}
\IfFileExists{microtype.sty}{% use microtype if available
  \usepackage[]{microtype}
  \UseMicrotypeSet[protrusion]{basicmath} % disable protrusion for tt fonts
}{}
\makeatletter
\@ifundefined{KOMAClassName}{% if non-KOMA class
  \IfFileExists{parskip.sty}{%
    \usepackage{parskip}
  }{% else
    \setlength{\parindent}{0pt}
    \setlength{\parskip}{6pt plus 2pt minus 1pt}}
}{% if KOMA class
  \KOMAoptions{parskip=half}}
\makeatother
\usepackage{xcolor}
\usepackage[margin=1in]{geometry}
\usepackage{color}
\usepackage{fancyvrb}
\newcommand{\VerbBar}{|}
\newcommand{\VERB}{\Verb[commandchars=\\\{\}]}
\DefineVerbatimEnvironment{Highlighting}{Verbatim}{commandchars=\\\{\}}
% Add ',fontsize=\small' for more characters per line
\usepackage{framed}
\definecolor{shadecolor}{RGB}{248,248,248}
\newenvironment{Shaded}{\begin{snugshade}}{\end{snugshade}}
\newcommand{\AlertTok}[1]{\textcolor[rgb]{0.94,0.16,0.16}{#1}}
\newcommand{\AnnotationTok}[1]{\textcolor[rgb]{0.56,0.35,0.01}{\textbf{\textit{#1}}}}
\newcommand{\AttributeTok}[1]{\textcolor[rgb]{0.13,0.29,0.53}{#1}}
\newcommand{\BaseNTok}[1]{\textcolor[rgb]{0.00,0.00,0.81}{#1}}
\newcommand{\BuiltInTok}[1]{#1}
\newcommand{\CharTok}[1]{\textcolor[rgb]{0.31,0.60,0.02}{#1}}
\newcommand{\CommentTok}[1]{\textcolor[rgb]{0.56,0.35,0.01}{\textit{#1}}}
\newcommand{\CommentVarTok}[1]{\textcolor[rgb]{0.56,0.35,0.01}{\textbf{\textit{#1}}}}
\newcommand{\ConstantTok}[1]{\textcolor[rgb]{0.56,0.35,0.01}{#1}}
\newcommand{\ControlFlowTok}[1]{\textcolor[rgb]{0.13,0.29,0.53}{\textbf{#1}}}
\newcommand{\DataTypeTok}[1]{\textcolor[rgb]{0.13,0.29,0.53}{#1}}
\newcommand{\DecValTok}[1]{\textcolor[rgb]{0.00,0.00,0.81}{#1}}
\newcommand{\DocumentationTok}[1]{\textcolor[rgb]{0.56,0.35,0.01}{\textbf{\textit{#1}}}}
\newcommand{\ErrorTok}[1]{\textcolor[rgb]{0.64,0.00,0.00}{\textbf{#1}}}
\newcommand{\ExtensionTok}[1]{#1}
\newcommand{\FloatTok}[1]{\textcolor[rgb]{0.00,0.00,0.81}{#1}}
\newcommand{\FunctionTok}[1]{\textcolor[rgb]{0.13,0.29,0.53}{\textbf{#1}}}
\newcommand{\ImportTok}[1]{#1}
\newcommand{\InformationTok}[1]{\textcolor[rgb]{0.56,0.35,0.01}{\textbf{\textit{#1}}}}
\newcommand{\KeywordTok}[1]{\textcolor[rgb]{0.13,0.29,0.53}{\textbf{#1}}}
\newcommand{\NormalTok}[1]{#1}
\newcommand{\OperatorTok}[1]{\textcolor[rgb]{0.81,0.36,0.00}{\textbf{#1}}}
\newcommand{\OtherTok}[1]{\textcolor[rgb]{0.56,0.35,0.01}{#1}}
\newcommand{\PreprocessorTok}[1]{\textcolor[rgb]{0.56,0.35,0.01}{\textit{#1}}}
\newcommand{\RegionMarkerTok}[1]{#1}
\newcommand{\SpecialCharTok}[1]{\textcolor[rgb]{0.81,0.36,0.00}{\textbf{#1}}}
\newcommand{\SpecialStringTok}[1]{\textcolor[rgb]{0.31,0.60,0.02}{#1}}
\newcommand{\StringTok}[1]{\textcolor[rgb]{0.31,0.60,0.02}{#1}}
\newcommand{\VariableTok}[1]{\textcolor[rgb]{0.00,0.00,0.00}{#1}}
\newcommand{\VerbatimStringTok}[1]{\textcolor[rgb]{0.31,0.60,0.02}{#1}}
\newcommand{\WarningTok}[1]{\textcolor[rgb]{0.56,0.35,0.01}{\textbf{\textit{#1}}}}
\usepackage{graphicx}
\makeatletter
\def\maxwidth{\ifdim\Gin@nat@width>\linewidth\linewidth\else\Gin@nat@width\fi}
\def\maxheight{\ifdim\Gin@nat@height>\textheight\textheight\else\Gin@nat@height\fi}
\makeatother
% Scale images if necessary, so that they will not overflow the page
% margins by default, and it is still possible to overwrite the defaults
% using explicit options in \includegraphics[width, height, ...]{}
\setkeys{Gin}{width=\maxwidth,height=\maxheight,keepaspectratio}
% Set default figure placement to htbp
\makeatletter
\def\fps@figure{htbp}
\makeatother
\setlength{\emergencystretch}{3em} % prevent overfull lines
\providecommand{\tightlist}{%
  \setlength{\itemsep}{0pt}\setlength{\parskip}{0pt}}
\setcounter{secnumdepth}{-\maxdimen} % remove section numbering
\ifLuaTeX
  \usepackage{selnolig}  % disable illegal ligatures
\fi
\usepackage{bookmark}
\IfFileExists{xurl.sty}{\usepackage{xurl}}{} % add URL line breaks if available
\urlstyle{same}
\hypersetup{
  pdftitle={Pokémon Dataset Analysis \& Legendary Prediction},
  pdfauthor={William L Shepherd},
  hidelinks,
  pdfcreator={LaTeX via pandoc}}

\title{Pokémon Dataset Analysis \& Legendary Prediction}
\author{William L Shepherd}
\date{August 12, 2023}

\begin{document}
\maketitle

\subsection{1. Introduction}\label{introduction}

\begin{Shaded}
\begin{Highlighting}[]
\NormalTok{knitr}\SpecialCharTok{::}\FunctionTok{include\_graphics}\NormalTok{(}\StringTok{"hero{-}img{-}3980048463.png"}\NormalTok{)}
\end{Highlighting}
\end{Shaded}

\includegraphics{hero-img-3980048463.png} This is a simple project which
I undertook last year (based on a DataCamps project), which aims to
introduce simple data analysis and machine learning concepts using a
data set of one of my favorite video-game franchises. The key objectives
of the project are the following:

\begin{itemize}
\item
  To develop foundation skills in data cleaning, analysis and
  visualization.
\item
  To familiarize myself with the R language.
\item
  To develop simple machine learning models utilizing the Random Forest
  and Decision Tree algorithms in order to predict whether Pokémon are
  legendary or not.
\end{itemize}

Hyperlink to data set:
\url{https://www.kaggle.com/datasets/rounakbanik/pokemon}

Hyperlink to banner picture:
\url{https://pokemonletsgo.pokemon.com/en-gb/how-to-play/}

\subsection{2. Implementation}\label{implementation}

\subsubsection{2.1 Loading Libraries}\label{loading-libraries}

\begin{Shaded}
\begin{Highlighting}[]
\FunctionTok{library}\NormalTok{(datasets) }
\FunctionTok{library}\NormalTok{(}\StringTok{"tidyverse"}\NormalTok{)}
\end{Highlighting}
\end{Shaded}

\begin{verbatim}
## -- Attaching core tidyverse packages ------------------------ tidyverse 2.0.0 --
## v dplyr     1.1.4     v readr     2.1.5
## v forcats   1.0.0     v stringr   1.5.1
## v ggplot2   3.5.1     v tibble    3.2.1
## v lubridate 1.9.3     v tidyr     1.3.1
## v purrr     1.0.2     
## -- Conflicts ------------------------------------------ tidyverse_conflicts() --
## x dplyr::filter() masks stats::filter()
## x dplyr::lag()    masks stats::lag()
## i Use the conflicted package (<http://conflicted.r-lib.org/>) to force all conflicts to become errors
\end{verbatim}

\begin{Shaded}
\begin{Highlighting}[]
\FunctionTok{library}\NormalTok{(ggplot2)}
\FunctionTok{library}\NormalTok{(rgl)}
\FunctionTok{library}\NormalTok{(plotly)}
\end{Highlighting}
\end{Shaded}

\begin{verbatim}
## 
## Attaching package: 'plotly'
## 
## The following object is masked from 'package:ggplot2':
## 
##     last_plot
## 
## The following object is masked from 'package:stats':
## 
##     filter
## 
## The following object is masked from 'package:graphics':
## 
##     layout
\end{verbatim}

\begin{Shaded}
\begin{Highlighting}[]
\FunctionTok{library}\NormalTok{(dplyr)}
\FunctionTok{library}\NormalTok{(gridExtra)}
\end{Highlighting}
\end{Shaded}

\begin{verbatim}
## 
## Attaching package: 'gridExtra'
## 
## The following object is masked from 'package:dplyr':
## 
##     combine
\end{verbatim}

\begin{Shaded}
\begin{Highlighting}[]
\FunctionTok{library}\NormalTok{(RColorBrewer)}
\FunctionTok{library}\NormalTok{(ggrepel)}
\FunctionTok{library}\NormalTok{(caret)}
\end{Highlighting}
\end{Shaded}

\begin{verbatim}
## Loading required package: lattice
## 
## Attaching package: 'caret'
## 
## The following object is masked from 'package:purrr':
## 
##     lift
\end{verbatim}

\begin{Shaded}
\begin{Highlighting}[]
\FunctionTok{library}\NormalTok{(e1071)}
\FunctionTok{library}\NormalTok{(randomForest)}
\end{Highlighting}
\end{Shaded}

\begin{verbatim}
## randomForest 4.7-1.2
## Type rfNews() to see new features/changes/bug fixes.
## 
## Attaching package: 'randomForest'
## 
## The following object is masked from 'package:gridExtra':
## 
##     combine
## 
## The following object is masked from 'package:dplyr':
## 
##     combine
## 
## The following object is masked from 'package:ggplot2':
## 
##     margin
\end{verbatim}

\begin{Shaded}
\begin{Highlighting}[]
\FunctionTok{library}\NormalTok{(tree)}
\FunctionTok{library}\NormalTok{(pROC)}
\end{Highlighting}
\end{Shaded}

\begin{verbatim}
## Type 'citation("pROC")' for a citation.
## 
## Attaching package: 'pROC'
## 
## The following objects are masked from 'package:stats':
## 
##     cov, smooth, var
\end{verbatim}

\begin{Shaded}
\begin{Highlighting}[]
\FunctionTok{library}\NormalTok{(RColorBrewer)}
\end{Highlighting}
\end{Shaded}

\subsubsection{2.2 Importing Data}\label{importing-data}

\begin{Shaded}
\begin{Highlighting}[]
\NormalTok{df }\OtherTok{\textless{}{-}} \FunctionTok{read.csv}\NormalTok{(}\StringTok{"pokemon.csv"}\NormalTok{)}
\end{Highlighting}
\end{Shaded}

\subsubsection{2.3 Cleaning Data}\label{cleaning-data}

Here I'm selecting the most relevant metrics which I want to analyse,
alongside ones which will be used for the machine learning models, to
predict whether a Pokemon is legendary or not. I'm leaving out metrics
such as the following: capture\_rate, abilities, experience growth, base
happiness, percentage growth and type 2.

\begin{Shaded}
\begin{Highlighting}[]
\NormalTok{df }\OtherTok{\textless{}{-}}\NormalTok{ df }\SpecialCharTok{\%\textgreater{}\%}
  \FunctionTok{select}\NormalTok{(name, type1, attack, defense, height\_m, sp\_attack, sp\_defense, speed, weight\_kg, generation, is\_legendary, hp)}
\end{Highlighting}
\end{Shaded}

\subsubsection{2.4 Displaying the Dataset's
Head}\label{displaying-the-datasets-head}

\begin{Shaded}
\begin{Highlighting}[]
\FunctionTok{head}\NormalTok{(df)}
\end{Highlighting}
\end{Shaded}

\begin{verbatim}
##         name type1 attack defense height_m sp_attack sp_defense speed weight_kg
## 1  Bulbasaur grass     49      49      0.7        65         65    45       6.9
## 2    Ivysaur grass     62      63      1.0        80         80    60      13.0
## 3   Venusaur grass    100     123      2.0       122        120    80     100.0
## 4 Charmander  fire     52      43      0.6        60         50    65       8.5
## 5 Charmeleon  fire     64      58      1.1        80         65    80      19.0
## 6  Charizard  fire    104      78      1.7       159        115   100      90.5
##   generation is_legendary hp
## 1          1            0 45
## 2          1            0 60
## 3          1            0 80
## 4          1            0 39
## 5          1            0 58
## 6          1            0 78
\end{verbatim}

\subsubsection{2.5 Data Analysis}\label{data-analysis}

\paragraph{2.5.1 Displaying the Number of Pokemon per
Generation}\label{displaying-the-number-of-pokemon-per-generation}

Generations in Pokémon simply refers to what generation of games a
particular Pokémon was introduced in (e.g.~Pikachu was introduced in
Generation I, whilst Dialga was in IV). The generation with the most
legendary Pokémon is the seventh, which has 17.

\begin{Shaded}
\begin{Highlighting}[]
\NormalTok{pokemon\_by\_gen }\OtherTok{\textless{}{-}}\NormalTok{ df }\SpecialCharTok{\%\textgreater{}\%}
  \FunctionTok{group\_by}\NormalTok{(generation) }\SpecialCharTok{\%\textgreater{}\%}
  \FunctionTok{summarize}\NormalTok{(}\AttributeTok{Count =} \FunctionTok{n}\NormalTok{())}

\NormalTok{colors }\OtherTok{\textless{}{-}} \FunctionTok{c}\NormalTok{(}\StringTok{\textquotesingle{}rgba(93, 164, 214, 0.8)\textquotesingle{}}\NormalTok{, }\StringTok{\textquotesingle{}rgba(255, 144, 14, 0.8)\textquotesingle{}}\NormalTok{, }\StringTok{\textquotesingle{}rgba(44, 160, 101, 0.8)\textquotesingle{}}\NormalTok{, }
            \StringTok{\textquotesingle{}rgba(255, 65, 54, 0.8)\textquotesingle{}}\NormalTok{, }\StringTok{\textquotesingle{}rgba(207, 114, 255, 0.8)\textquotesingle{}}\NormalTok{, }\StringTok{\textquotesingle{}rgba(127, 96, 0, 0.8)\textquotesingle{}}\NormalTok{)}

\FunctionTok{plot\_ly}\NormalTok{(pokemon\_by\_gen, }\AttributeTok{x =} \SpecialCharTok{\textasciitilde{}}\NormalTok{generation, }\AttributeTok{y =} \SpecialCharTok{\textasciitilde{}}\NormalTok{Count,}
        \AttributeTok{type =} \StringTok{\textquotesingle{}bar\textquotesingle{}}\NormalTok{,}
        \AttributeTok{marker =} \FunctionTok{list}\NormalTok{(}\AttributeTok{color =}\NormalTok{ colors, }\AttributeTok{opacity =} \FloatTok{0.8}\NormalTok{),}
        \AttributeTok{text =} \SpecialCharTok{\textasciitilde{}}\FunctionTok{paste}\NormalTok{(}\StringTok{\textquotesingle{}Generation:\textquotesingle{}}\NormalTok{, generation, }\StringTok{\textquotesingle{}Count:\textquotesingle{}}\NormalTok{, Count),}
        \AttributeTok{hoverinfo =} \StringTok{\textquotesingle{}text\textquotesingle{}}\NormalTok{,}
        \AttributeTok{textposition =} \StringTok{\textquotesingle{}none\textquotesingle{}}\NormalTok{) }\SpecialCharTok{\%\textgreater{}\%}
  
\FunctionTok{layout}\NormalTok{(}\AttributeTok{title =} \FunctionTok{list}\NormalTok{(}\AttributeTok{text =} \StringTok{\textquotesingle{}Pokémon per Generation\textquotesingle{}}\NormalTok{, }\AttributeTok{font =} \FunctionTok{list}\NormalTok{(}\AttributeTok{size =} \DecValTok{24}\NormalTok{)),}
       \AttributeTok{xaxis =} \FunctionTok{list}\NormalTok{(}\AttributeTok{title =} \StringTok{\textquotesingle{}Generation\textquotesingle{}}\NormalTok{, }\AttributeTok{titlefont =} \FunctionTok{list}\NormalTok{(}\AttributeTok{size =} \DecValTok{18}\NormalTok{), }\AttributeTok{tickfont =} \FunctionTok{list}\NormalTok{(}\AttributeTok{size =} \DecValTok{14}\NormalTok{)),}
       \AttributeTok{yaxis =} \FunctionTok{list}\NormalTok{(}\AttributeTok{title =} \StringTok{\textquotesingle{}Count\textquotesingle{}}\NormalTok{, }\AttributeTok{titlefont =} \FunctionTok{list}\NormalTok{(}\AttributeTok{size =} \DecValTok{18}\NormalTok{), }\AttributeTok{tickfont =} \FunctionTok{list}\NormalTok{(}\AttributeTok{size =} \DecValTok{14}\NormalTok{)),}
       \AttributeTok{plot\_bgcolor =} \StringTok{\textquotesingle{}rgba(245, 246, 249, 1)\textquotesingle{}}\NormalTok{,}
       \AttributeTok{paper\_bgcolor =} \StringTok{\textquotesingle{}rgba(245, 246, 249, 1)\textquotesingle{}}\NormalTok{,}
       \AttributeTok{margin =} \FunctionTok{list}\NormalTok{(}\AttributeTok{l =} \DecValTok{50}\NormalTok{, }\AttributeTok{r =} \DecValTok{50}\NormalTok{, }\AttributeTok{b =} \DecValTok{100}\NormalTok{, }\AttributeTok{t =} \DecValTok{100}\NormalTok{, }\AttributeTok{pad =} \DecValTok{4}\NormalTok{))}
\end{Highlighting}
\end{Shaded}

\begin{Shaded}
\begin{Highlighting}[]
\NormalTok{pokemon\_by\_gen\_legendary }\OtherTok{\textless{}{-}}\NormalTok{ df }\SpecialCharTok{\%\textgreater{}\%}
  \FunctionTok{filter}\NormalTok{(is\_legendary }\SpecialCharTok{==} \ConstantTok{TRUE}\NormalTok{) }\SpecialCharTok{\%\textgreater{}\%}
  \FunctionTok{group\_by}\NormalTok{(generation) }\SpecialCharTok{\%\textgreater{}\%}
  \FunctionTok{summarize}\NormalTok{(}\AttributeTok{Count =} \FunctionTok{n}\NormalTok{())}

\FunctionTok{plot\_ly}\NormalTok{(pokemon\_by\_gen\_legendary, }\AttributeTok{x =} \SpecialCharTok{\textasciitilde{}}\NormalTok{generation, }\AttributeTok{y =} \SpecialCharTok{\textasciitilde{}}\NormalTok{Count,}
        \AttributeTok{type =} \StringTok{\textquotesingle{}bar\textquotesingle{}}\NormalTok{,}
        \AttributeTok{marker =} \FunctionTok{list}\NormalTok{(}\AttributeTok{color =}\NormalTok{ colors, }\AttributeTok{opacity =} \FloatTok{0.8}\NormalTok{),}
        \AttributeTok{text =} \SpecialCharTok{\textasciitilde{}}\FunctionTok{paste}\NormalTok{(}\StringTok{\textquotesingle{}Generation:\textquotesingle{}}\NormalTok{, generation, }\StringTok{\textquotesingle{}Count:\textquotesingle{}}\NormalTok{, Count),}
        \AttributeTok{hoverinfo =} \StringTok{\textquotesingle{}text\textquotesingle{}}\NormalTok{,}
        \AttributeTok{textposition =} \StringTok{\textquotesingle{}none\textquotesingle{}}\NormalTok{) }\SpecialCharTok{\%\textgreater{}\%}
  
\FunctionTok{layout}\NormalTok{(}\AttributeTok{title =} \FunctionTok{list}\NormalTok{(}\AttributeTok{text =} \StringTok{\textquotesingle{}Legendary Pokémon per Generation\textquotesingle{}}\NormalTok{, }\AttributeTok{font =} \FunctionTok{list}\NormalTok{(}\AttributeTok{size =} \DecValTok{24}\NormalTok{)),}
       \AttributeTok{xaxis =} \FunctionTok{list}\NormalTok{(}\AttributeTok{title =} \StringTok{\textquotesingle{}Generation\textquotesingle{}}\NormalTok{, }\AttributeTok{titlefont =} \FunctionTok{list}\NormalTok{(}\AttributeTok{size =} \DecValTok{18}\NormalTok{), }\AttributeTok{tickfont =} \FunctionTok{list}\NormalTok{(}\AttributeTok{size =} \DecValTok{14}\NormalTok{)),}
       \AttributeTok{yaxis =} \FunctionTok{list}\NormalTok{(}\AttributeTok{title =} \StringTok{\textquotesingle{}Count\textquotesingle{}}\NormalTok{, }\AttributeTok{titlefont =} \FunctionTok{list}\NormalTok{(}\AttributeTok{size =} \DecValTok{18}\NormalTok{), }\AttributeTok{tickfont =} \FunctionTok{list}\NormalTok{(}\AttributeTok{size =} \DecValTok{14}\NormalTok{)),}
       \AttributeTok{plot\_bgcolor =} \StringTok{\textquotesingle{}rgba(245, 246, 249, 1)\textquotesingle{}}\NormalTok{,}
       \AttributeTok{paper\_bgcolor =} \StringTok{\textquotesingle{}rgba(245, 246, 249, 1)\textquotesingle{}}\NormalTok{,}
       \AttributeTok{margin =} \FunctionTok{list}\NormalTok{(}\AttributeTok{l =} \DecValTok{50}\NormalTok{, }\AttributeTok{r =} \DecValTok{50}\NormalTok{, }\AttributeTok{b =} \DecValTok{100}\NormalTok{, }\AttributeTok{t =} \DecValTok{100}\NormalTok{, }\AttributeTok{pad =} \DecValTok{4}\NormalTok{))}
\end{Highlighting}
\end{Shaded}

\paragraph{2.5.2 Displaying the Number of Pokemon per
Type}\label{displaying-the-number-of-pokemon-per-type}

As shown by the graph below, the most common type of legendary Pokemon
is psychic, with as of generation 7, the absence of poison or fighting
legends.

\begin{Shaded}
\begin{Highlighting}[]
\NormalTok{pokemon\_by\_type }\OtherTok{\textless{}{-}}\NormalTok{ df }\SpecialCharTok{\%\textgreater{}\%}
  \FunctionTok{group\_by}\NormalTok{(type1) }\SpecialCharTok{\%\textgreater{}\%}
  \FunctionTok{summarize}\NormalTok{(}\AttributeTok{Count =} \FunctionTok{n}\NormalTok{())}

\FunctionTok{plot\_ly}\NormalTok{(pokemon\_by\_type, }\AttributeTok{x =} \SpecialCharTok{\textasciitilde{}}\NormalTok{type1, }\AttributeTok{y =} \SpecialCharTok{\textasciitilde{}}\NormalTok{Count,}
        \AttributeTok{type =} \StringTok{\textquotesingle{}bar\textquotesingle{}}\NormalTok{,}
        \AttributeTok{marker =} \FunctionTok{list}\NormalTok{(}\AttributeTok{color =} \SpecialCharTok{\textasciitilde{}}\NormalTok{Count, }\AttributeTok{colorscale =} \StringTok{\textquotesingle{}Viridis\textquotesingle{}}\NormalTok{, }\AttributeTok{opacity =} \FloatTok{0.8}\NormalTok{),}
        \AttributeTok{text =} \SpecialCharTok{\textasciitilde{}}\FunctionTok{paste}\NormalTok{(}\StringTok{\textquotesingle{}Type:\textquotesingle{}}\NormalTok{, type1, }\StringTok{\textquotesingle{}\textless{}br\textgreater{}Count:\textquotesingle{}}\NormalTok{, Count),}
        \AttributeTok{hoverinfo =} \StringTok{\textquotesingle{}text\textquotesingle{}}\NormalTok{,}
        \AttributeTok{textposition =} \StringTok{\textquotesingle{}none\textquotesingle{}}\NormalTok{) }\SpecialCharTok{\%\textgreater{}\%}

\FunctionTok{layout}\NormalTok{(}\AttributeTok{title =} \FunctionTok{list}\NormalTok{(}\AttributeTok{text =} \StringTok{\textquotesingle{}Pokémon by Type\textquotesingle{}}\NormalTok{, }\AttributeTok{font =} \FunctionTok{list}\NormalTok{(}\AttributeTok{size =} \DecValTok{24}\NormalTok{)),}
       \AttributeTok{xaxis =} \FunctionTok{list}\NormalTok{(}\AttributeTok{title =} \StringTok{\textquotesingle{}Type\textquotesingle{}}\NormalTok{, }\AttributeTok{titlefont =} \FunctionTok{list}\NormalTok{(}\AttributeTok{size =} \DecValTok{18}\NormalTok{), }\AttributeTok{tickfont =} \FunctionTok{list}\NormalTok{(}\AttributeTok{size =} \DecValTok{14}\NormalTok{)),}
       \AttributeTok{yaxis =} \FunctionTok{list}\NormalTok{(}\AttributeTok{title =} \StringTok{\textquotesingle{}Count\textquotesingle{}}\NormalTok{, }\AttributeTok{titlefont =} \FunctionTok{list}\NormalTok{(}\AttributeTok{size =} \DecValTok{18}\NormalTok{), }\AttributeTok{tickfont =} \FunctionTok{list}\NormalTok{(}\AttributeTok{size =} \DecValTok{14}\NormalTok{)),}
       \AttributeTok{plot\_bgcolor =} \StringTok{\textquotesingle{}rgba(245, 246, 249, 1)\textquotesingle{}}\NormalTok{,}
       \AttributeTok{paper\_bgcolor =} \StringTok{\textquotesingle{}rgba(245, 246, 249, 1)\textquotesingle{}}\NormalTok{,}
       \AttributeTok{margin =} \FunctionTok{list}\NormalTok{(}\AttributeTok{l =} \DecValTok{50}\NormalTok{, }\AttributeTok{r =} \DecValTok{50}\NormalTok{, }\AttributeTok{b =} \DecValTok{100}\NormalTok{, }\AttributeTok{t =} \DecValTok{100}\NormalTok{, }\AttributeTok{pad =} \DecValTok{4}\NormalTok{))}
\end{Highlighting}
\end{Shaded}

\begin{Shaded}
\begin{Highlighting}[]
\NormalTok{pokemon\_by\_type }\OtherTok{\textless{}{-}}\NormalTok{ df }\SpecialCharTok{\%\textgreater{}\%}
  \FunctionTok{filter}\NormalTok{(is\_legendary }\SpecialCharTok{==} \DecValTok{1}\NormalTok{) }\SpecialCharTok{\%\textgreater{}\%}
  \FunctionTok{group\_by}\NormalTok{(type1) }\SpecialCharTok{\%\textgreater{}\%}
  \FunctionTok{summarize}\NormalTok{(}\AttributeTok{Count =} \FunctionTok{n}\NormalTok{())}

\CommentTok{\# Rest of the code remains the same}
\FunctionTok{plot\_ly}\NormalTok{(pokemon\_by\_type, }\AttributeTok{x =} \SpecialCharTok{\textasciitilde{}}\NormalTok{type1, }\AttributeTok{y =} \SpecialCharTok{\textasciitilde{}}\NormalTok{Count,}
        \AttributeTok{type =} \StringTok{\textquotesingle{}bar\textquotesingle{}}\NormalTok{,}
        \AttributeTok{marker =} \FunctionTok{list}\NormalTok{(}\AttributeTok{color =} \SpecialCharTok{\textasciitilde{}}\NormalTok{Count, }\AttributeTok{colorscale =} \StringTok{\textquotesingle{}Viridis\textquotesingle{}}\NormalTok{, }\AttributeTok{opacity =} \FloatTok{0.8}\NormalTok{),}
        \AttributeTok{text =} \SpecialCharTok{\textasciitilde{}}\FunctionTok{paste}\NormalTok{(}\StringTok{\textquotesingle{}Type:\textquotesingle{}}\NormalTok{, type1, }\StringTok{\textquotesingle{}\textless{}br\textgreater{}Count:\textquotesingle{}}\NormalTok{, Count),}
        \AttributeTok{hoverinfo =} \StringTok{\textquotesingle{}text\textquotesingle{}}\NormalTok{,}
        \AttributeTok{textposition =} \StringTok{\textquotesingle{}none\textquotesingle{}}\NormalTok{) }\SpecialCharTok{\%\textgreater{}\%}

\FunctionTok{layout}\NormalTok{(}\AttributeTok{title =} \FunctionTok{list}\NormalTok{(}\AttributeTok{text =} \StringTok{\textquotesingle{}Legendary Pokémon by Type\textquotesingle{}}\NormalTok{, }\AttributeTok{font =} \FunctionTok{list}\NormalTok{(}\AttributeTok{size =} \DecValTok{24}\NormalTok{)),}
       \AttributeTok{xaxis =} \FunctionTok{list}\NormalTok{(}\AttributeTok{title =} \StringTok{\textquotesingle{}Type\textquotesingle{}}\NormalTok{, }\AttributeTok{titlefont =} \FunctionTok{list}\NormalTok{(}\AttributeTok{size =} \DecValTok{18}\NormalTok{), }\AttributeTok{tickfont =} \FunctionTok{list}\NormalTok{(}\AttributeTok{size =} \DecValTok{14}\NormalTok{)),}
       \AttributeTok{yaxis =} \FunctionTok{list}\NormalTok{(}\AttributeTok{title =} \StringTok{\textquotesingle{}Count\textquotesingle{}}\NormalTok{, }\AttributeTok{titlefont =} \FunctionTok{list}\NormalTok{(}\AttributeTok{size =} \DecValTok{18}\NormalTok{), }\AttributeTok{tickfont =} \FunctionTok{list}\NormalTok{(}\AttributeTok{size =} \DecValTok{14}\NormalTok{)),}
       \AttributeTok{plot\_bgcolor =} \StringTok{\textquotesingle{}rgba(245, 246, 249, 1)\textquotesingle{}}\NormalTok{,}
       \AttributeTok{paper\_bgcolor =} \StringTok{\textquotesingle{}rgba(245, 246, 249, 1)\textquotesingle{}}\NormalTok{,}
       \AttributeTok{margin =} \FunctionTok{list}\NormalTok{(}\AttributeTok{l =} \DecValTok{50}\NormalTok{, }\AttributeTok{r =} \DecValTok{50}\NormalTok{, }\AttributeTok{b =} \DecValTok{100}\NormalTok{, }\AttributeTok{t =} \DecValTok{100}\NormalTok{, }\AttributeTok{pad =} \DecValTok{4}\NormalTok{))}
\end{Highlighting}
\end{Shaded}

\paragraph{2.5.3 Legendary Pokemon
Analysis}\label{legendary-pokemon-analysis}

\subparagraph{2.5.3.1 What Percentage of Pokemon are
Legendary?}\label{what-percentage-of-pokemon-are-legendary}

\begin{Shaded}
\begin{Highlighting}[]
\NormalTok{num\_legendary\_pokemon }\OtherTok{\textless{}{-}}\NormalTok{ df }\SpecialCharTok{\%\textgreater{}\%}
  \FunctionTok{filter}\NormalTok{(is\_legendary }\SpecialCharTok{==} \DecValTok{1}\NormalTok{) }\SpecialCharTok{\%\textgreater{}\%}
  \FunctionTok{nrow}\NormalTok{()}

\NormalTok{legendary\_pokemon }\OtherTok{\textless{}{-}}\NormalTok{ df }\SpecialCharTok{\%\textgreater{}\%}
  \FunctionTok{filter}\NormalTok{(is\_legendary }\SpecialCharTok{==} \DecValTok{1}\NormalTok{)}

\NormalTok{pokemon\_type }\OtherTok{\textless{}{-}}\NormalTok{ df }\SpecialCharTok{\%\textgreater{}\%}
  \FunctionTok{group\_by}\NormalTok{(is\_legendary) }\SpecialCharTok{\%\textgreater{}\%}
  \FunctionTok{summarize}\NormalTok{(}\AttributeTok{count =} \FunctionTok{n}\NormalTok{())}

\FunctionTok{plot\_ly}\NormalTok{(pokemon\_type, }\AttributeTok{labels =} \SpecialCharTok{\textasciitilde{}}\FunctionTok{factor}\NormalTok{(is\_legendary), }\AttributeTok{values =} \SpecialCharTok{\textasciitilde{}}\NormalTok{count, }\AttributeTok{type =} \StringTok{\textquotesingle{}pie\textquotesingle{}}\NormalTok{,}
        \AttributeTok{marker =} \FunctionTok{list}\NormalTok{(}\AttributeTok{colors =} \FunctionTok{c}\NormalTok{(}\StringTok{\textquotesingle{}purple\textquotesingle{}}\NormalTok{, }\StringTok{\textquotesingle{}yellow\textquotesingle{}}\NormalTok{)),}
        \AttributeTok{textinfo =} \StringTok{\textquotesingle{}label+value+percent\textquotesingle{}}\NormalTok{) }\SpecialCharTok{\%\textgreater{}\%}
        \FunctionTok{layout}\NormalTok{(}\AttributeTok{title =} \StringTok{\textquotesingle{}\textquotesingle{}}\NormalTok{)}
\end{Highlighting}
\end{Shaded}

\subparagraph{2.5.3.2 Height vs Weight}\label{height-vs-weight}

As shown by the graph below, which explores the relationship between
height (m) and weight (kg); legendary Pokémon compared to their regular
counterparts are generally much heavier and taller with some exceptions
including: Onix, Steelix and Wailord.

\begin{Shaded}
\begin{Highlighting}[]
\NormalTok{top\_5\_heaviest }\OtherTok{\textless{}{-}}\NormalTok{ df }\SpecialCharTok{\%\textgreater{}\%}
  \FunctionTok{top\_n}\NormalTok{(}\DecValTok{5}\NormalTok{, weight\_kg)}

\NormalTok{top\_5\_tallest }\OtherTok{\textless{}{-}}\NormalTok{ df }\SpecialCharTok{\%\textgreater{}\%}
  \FunctionTok{top\_n}\NormalTok{(}\DecValTok{5}\NormalTok{, height\_m)}

\NormalTok{combined\_top\_5 }\OtherTok{\textless{}{-}} \FunctionTok{rbind}\NormalTok{(top\_5\_heaviest, top\_5\_tallest)}

\FunctionTok{ggplot}\NormalTok{(df, }\FunctionTok{aes}\NormalTok{(}\AttributeTok{x =}\NormalTok{ height\_m, }\AttributeTok{y =}\NormalTok{ weight\_kg, }\AttributeTok{color =} \FunctionTok{factor}\NormalTok{(is\_legendary))) }\SpecialCharTok{+}
  \FunctionTok{geom\_point}\NormalTok{() }\SpecialCharTok{+}
  \FunctionTok{geom\_text\_repel}\NormalTok{(}\AttributeTok{data =}\NormalTok{ combined\_top\_5, }\FunctionTok{aes}\NormalTok{(}\AttributeTok{label =}\NormalTok{ name), }\AttributeTok{size =} \DecValTok{3}\NormalTok{, }\AttributeTok{nudge\_x =} \FloatTok{0.2}\NormalTok{, }\AttributeTok{nudge\_y =} \FloatTok{0.2}\NormalTok{) }\SpecialCharTok{+}
  \FunctionTok{labs}\NormalTok{(}\AttributeTok{title =} \StringTok{""}\NormalTok{, }\AttributeTok{x =} \StringTok{"Height (meters)"}\NormalTok{, }\AttributeTok{y =} \StringTok{"Weight (kilograms)"}\NormalTok{) }\SpecialCharTok{+}
  \FunctionTok{scale\_color\_manual}\NormalTok{(}\AttributeTok{values =} \FunctionTok{c}\NormalTok{(}\StringTok{"1"} \OtherTok{=} \StringTok{"blue"}\NormalTok{, }\StringTok{"0"} \OtherTok{=} \StringTok{"red"}\NormalTok{), }\AttributeTok{labels =} \FunctionTok{c}\NormalTok{(}\StringTok{"Regular"}\NormalTok{, }\StringTok{"Legendary"}\NormalTok{))}
\end{Highlighting}
\end{Shaded}

\begin{verbatim}
## Warning: Removed 20 rows containing missing values or values outside the scale range
## (`geom_point()`).
\end{verbatim}

\includegraphics{pokemon_files/figure-latex/unnamed-chunk-11-1.pdf}

\subparagraph{2.5.3.3 Comparing Speed Against Other
Metrics}\label{comparing-speed-against-other-metrics}

The below graphs illustrate the relationship between Speed and various
other attributes (Attack, Defense, Height, Weight, Special Attack,
Special Defense) for Regular and Legendary Pokémon. Each graph consists
of a scatter plot where the x-axis represents the attribute and the
y-axis represents Speed. The points are color-coded to distinguish
between Regular and Legendary Pokémon.

The General Trends include the following:

\begin{itemize}
\item
  Legendary Pokémon exhibit a strong positive correlation between speed
  and both attack and special attack for legendary Pokémon, thus
  suggesting Legendary Pokémon with higher attack or special attack
  values tend to have correspondingly higher speed.
\item
  A weak negative correlation is evident between speed and both defense
  and special defense for both regular and legendary Pokémon; indicating
  that those with higher defence stats generally have lower speed.
\end{itemize}

\begin{Shaded}
\begin{Highlighting}[]
\NormalTok{speed\_vs\_attack }\OtherTok{\textless{}{-}} \FunctionTok{ggplot}\NormalTok{(df, }\FunctionTok{aes}\NormalTok{(}\AttributeTok{x =}\NormalTok{ attack, }\AttributeTok{y =}\NormalTok{ speed, }\AttributeTok{color =} \FunctionTok{factor}\NormalTok{(is\_legendary))) }\SpecialCharTok{+}
  \FunctionTok{geom\_point}\NormalTok{() }\SpecialCharTok{+}
  \FunctionTok{labs}\NormalTok{(}\AttributeTok{title =} \StringTok{"Speed vs. Attack"}\NormalTok{, }\AttributeTok{x =} \StringTok{"Attack"}\NormalTok{, }\AttributeTok{y =} \StringTok{"Speed"}\NormalTok{) }\SpecialCharTok{+}
  \FunctionTok{scale\_color\_manual}\NormalTok{(}\AttributeTok{values =} \FunctionTok{c}\NormalTok{(}\StringTok{"1"} \OtherTok{=} \StringTok{"blue"}\NormalTok{, }\StringTok{"0"} \OtherTok{=} \StringTok{"red"}\NormalTok{), }\AttributeTok{labels =} \FunctionTok{c}\NormalTok{(}\StringTok{"Regular"}\NormalTok{, }\StringTok{"Legendary"}\NormalTok{))}

\NormalTok{speed\_vs\_defense }\OtherTok{\textless{}{-}} \FunctionTok{ggplot}\NormalTok{(df, }\FunctionTok{aes}\NormalTok{(}\AttributeTok{x =}\NormalTok{ defense, }\AttributeTok{y =}\NormalTok{ speed, }\AttributeTok{color =} \FunctionTok{factor}\NormalTok{(is\_legendary))) }\SpecialCharTok{+}
  \FunctionTok{geom\_point}\NormalTok{() }\SpecialCharTok{+}
  \FunctionTok{labs}\NormalTok{(}\AttributeTok{title =} \StringTok{"Speed vs. Defense"}\NormalTok{, }\AttributeTok{x =} \StringTok{"Defense"}\NormalTok{, }\AttributeTok{y =} \StringTok{"Speed"}\NormalTok{) }\SpecialCharTok{+}
  \FunctionTok{scale\_color\_manual}\NormalTok{(}\AttributeTok{values =} \FunctionTok{c}\NormalTok{(}\StringTok{"1"} \OtherTok{=} \StringTok{"blue"}\NormalTok{, }\StringTok{"0"} \OtherTok{=} \StringTok{"red"}\NormalTok{), }\AttributeTok{labels =} \FunctionTok{c}\NormalTok{(}\StringTok{"Regular"}\NormalTok{, }\StringTok{"Legendary"}\NormalTok{))}

\NormalTok{speed\_vs\_height }\OtherTok{\textless{}{-}} \FunctionTok{ggplot}\NormalTok{(df, }\FunctionTok{aes}\NormalTok{(}\AttributeTok{x =}\NormalTok{ height\_m, }\AttributeTok{y =}\NormalTok{ speed, }\AttributeTok{color =} \FunctionTok{factor}\NormalTok{(is\_legendary))) }\SpecialCharTok{+}
  \FunctionTok{geom\_point}\NormalTok{() }\SpecialCharTok{+}
  \FunctionTok{labs}\NormalTok{(}\AttributeTok{title =} \StringTok{"Speed vs. Height"}\NormalTok{, }\AttributeTok{x =} \StringTok{"Height"}\NormalTok{, }\AttributeTok{y =} \StringTok{"Speed"}\NormalTok{) }\SpecialCharTok{+}
  \FunctionTok{scale\_color\_manual}\NormalTok{(}\AttributeTok{values =} \FunctionTok{c}\NormalTok{(}\StringTok{"1"} \OtherTok{=} \StringTok{"blue"}\NormalTok{, }\StringTok{"0"} \OtherTok{=} \StringTok{"red"}\NormalTok{), }\AttributeTok{labels =} \FunctionTok{c}\NormalTok{(}\StringTok{"Regular"}\NormalTok{, }\StringTok{"Legendary"}\NormalTok{))}

\NormalTok{speed\_vs\_weight }\OtherTok{\textless{}{-}} \FunctionTok{ggplot}\NormalTok{(df, }\FunctionTok{aes}\NormalTok{(}\AttributeTok{x =}\NormalTok{ weight\_kg, }\AttributeTok{y =}\NormalTok{ speed, }\AttributeTok{color =} \FunctionTok{factor}\NormalTok{(is\_legendary))) }\SpecialCharTok{+}
  \FunctionTok{geom\_point}\NormalTok{() }\SpecialCharTok{+}
  \FunctionTok{labs}\NormalTok{(}\AttributeTok{title =} \StringTok{"Speed vs. Weight"}\NormalTok{, }\AttributeTok{x =} \StringTok{"Weight"}\NormalTok{, }\AttributeTok{y =} \StringTok{"Speed"}\NormalTok{) }\SpecialCharTok{+}
  \FunctionTok{scale\_color\_manual}\NormalTok{(}\AttributeTok{values =} \FunctionTok{c}\NormalTok{(}\StringTok{"1"} \OtherTok{=} \StringTok{"blue"}\NormalTok{, }\StringTok{"0"} \OtherTok{=} \StringTok{"red"}\NormalTok{), }\AttributeTok{labels =} \FunctionTok{c}\NormalTok{(}\StringTok{"Regular"}\NormalTok{, }\StringTok{"Legendary"}\NormalTok{))}

\NormalTok{speed\_vs\_spattack }\OtherTok{\textless{}{-}} \FunctionTok{ggplot}\NormalTok{(df, }\FunctionTok{aes}\NormalTok{(}\AttributeTok{x =}\NormalTok{ sp\_attack, }\AttributeTok{y =}\NormalTok{ speed, }\AttributeTok{color =} \FunctionTok{factor}\NormalTok{(is\_legendary))) }\SpecialCharTok{+}
  \FunctionTok{geom\_point}\NormalTok{() }\SpecialCharTok{+}
  \FunctionTok{labs}\NormalTok{(}\AttributeTok{title =} \StringTok{"Speed vs. Special Attack"}\NormalTok{, }\AttributeTok{x =} \StringTok{"Special Attack"}\NormalTok{, }\AttributeTok{y =} \StringTok{"Speed"}\NormalTok{) }\SpecialCharTok{+}
  \FunctionTok{scale\_color\_manual}\NormalTok{(}\AttributeTok{values =} \FunctionTok{c}\NormalTok{(}\StringTok{"1"} \OtherTok{=} \StringTok{"blue"}\NormalTok{, }\StringTok{"0"} \OtherTok{=} \StringTok{"red"}\NormalTok{), }\AttributeTok{labels =} \FunctionTok{c}\NormalTok{(}\StringTok{"Regular"}\NormalTok{, }\StringTok{"Legendary"}\NormalTok{))}

\NormalTok{speed\_vs\_spdefence }\OtherTok{\textless{}{-}} \FunctionTok{ggplot}\NormalTok{(df, }\FunctionTok{aes}\NormalTok{(}\AttributeTok{x =}\NormalTok{ sp\_attack, }\AttributeTok{y =}\NormalTok{ speed, }\AttributeTok{color =} \FunctionTok{factor}\NormalTok{(is\_legendary))) }\SpecialCharTok{+}
  \FunctionTok{geom\_point}\NormalTok{() }\SpecialCharTok{+}
  \FunctionTok{labs}\NormalTok{(}\AttributeTok{title =} \StringTok{"Speed vs. Special Defence"}\NormalTok{, }\AttributeTok{x =} \StringTok{"Special Defence"}\NormalTok{, }\AttributeTok{y =} \StringTok{"Speed"}\NormalTok{) }\SpecialCharTok{+}
  \FunctionTok{scale\_color\_manual}\NormalTok{(}\AttributeTok{values =} \FunctionTok{c}\NormalTok{(}\StringTok{"1"} \OtherTok{=} \StringTok{"blue"}\NormalTok{, }\StringTok{"0"} \OtherTok{=} \StringTok{"red"}\NormalTok{), }\AttributeTok{labels =} \FunctionTok{c}\NormalTok{(}\StringTok{"Regular"}\NormalTok{, }\StringTok{"Legendary"}\NormalTok{))}

\FunctionTok{grid.arrange}\NormalTok{(speed\_vs\_attack, speed\_vs\_defense, speed\_vs\_height, speed\_vs\_weight, speed\_vs\_spattack, speed\_vs\_spdefence, }\AttributeTok{ncol =} \DecValTok{2}\NormalTok{)}
\end{Highlighting}
\end{Shaded}

\begin{verbatim}
## Warning: Removed 20 rows containing missing values or values outside the scale range
## (`geom_point()`).
## Removed 20 rows containing missing values or values outside the scale range
## (`geom_point()`).
\end{verbatim}

\includegraphics{pokemon_files/figure-latex/unnamed-chunk-12-1.pdf}

\subparagraph{2.5.3.4 Boxplots of all
Metrics}\label{boxplots-of-all-metrics}

The following box plot diagrams illustrate the distribution of various
attributes for both regular and legendary Pokémon. Each plot consists of
two boxes, one for Regular Pokémon (red) and one for Legendary Pokémon
(blue). The boxes represent the interquartile range (IQR), while the
lines extending from the boxes indicate the range of the data excluding
outliers. The dots represent individual data points.

Overall, the plots suggest that Legendary Pokémon tend to have higher
values for most attributes compared to Regular Pokémon. This is
particularly evident for Attack, Special Attack, and Speed, where the
median and upper quartile of Legendary Pokémon are significantly higher.

\begin{Shaded}
\begin{Highlighting}[]
\NormalTok{attack }\OtherTok{\textless{}{-}} \FunctionTok{ggplot}\NormalTok{(}\FunctionTok{na.omit}\NormalTok{(df), }\FunctionTok{aes}\NormalTok{(}\AttributeTok{x =} \FunctionTok{factor}\NormalTok{(is\_legendary), }\AttributeTok{y =}\NormalTok{ attack, }\AttributeTok{fill =} \FunctionTok{factor}\NormalTok{(is\_legendary))) }\SpecialCharTok{+}
  \FunctionTok{geom\_boxplot}\NormalTok{() }\SpecialCharTok{+}
  \FunctionTok{labs}\NormalTok{(}\AttributeTok{title =} \StringTok{""}\NormalTok{, }\AttributeTok{x =} \StringTok{"Legendary Status"}\NormalTok{, }\AttributeTok{y =} \StringTok{"Attack"}\NormalTok{) }\SpecialCharTok{+}
  \FunctionTok{scale\_x\_discrete}\NormalTok{(}\AttributeTok{labels =} \FunctionTok{c}\NormalTok{(}\StringTok{"Regular"}\NormalTok{, }\StringTok{"Legendary"}\NormalTok{)) }\SpecialCharTok{+}
  \FunctionTok{scale\_fill\_manual}\NormalTok{(}\AttributeTok{values =} \FunctionTok{c}\NormalTok{(}\StringTok{"1"} \OtherTok{=} \StringTok{"blue"}\NormalTok{, }\StringTok{"0"} \OtherTok{=} \StringTok{"red"}\NormalTok{), }\AttributeTok{labels =} \FunctionTok{c}\NormalTok{(}\StringTok{"Regular"}\NormalTok{, }\StringTok{"Legendary"}\NormalTok{)) }\SpecialCharTok{+}
  \FunctionTok{theme\_bw}\NormalTok{()}

\NormalTok{defence }\OtherTok{\textless{}{-}} \FunctionTok{ggplot}\NormalTok{(}\FunctionTok{na.omit}\NormalTok{(df), }\FunctionTok{aes}\NormalTok{(}\AttributeTok{x =} \FunctionTok{factor}\NormalTok{(is\_legendary), }\AttributeTok{y =}\NormalTok{ defense, }\AttributeTok{fill =} \FunctionTok{factor}\NormalTok{(is\_legendary))) }\SpecialCharTok{+}
  \FunctionTok{geom\_boxplot}\NormalTok{() }\SpecialCharTok{+}
  \FunctionTok{labs}\NormalTok{(}\AttributeTok{title =} \StringTok{""}\NormalTok{, }\AttributeTok{x =} \StringTok{"Legendary Status"}\NormalTok{, }\AttributeTok{y =} \StringTok{"Defense"}\NormalTok{) }\SpecialCharTok{+}
  \FunctionTok{scale\_x\_discrete}\NormalTok{(}\AttributeTok{labels =} \FunctionTok{c}\NormalTok{(}\StringTok{"Regular"}\NormalTok{, }\StringTok{"Legendary"}\NormalTok{)) }\SpecialCharTok{+}
  \FunctionTok{scale\_fill\_manual}\NormalTok{(}\AttributeTok{values =} \FunctionTok{c}\NormalTok{(}\StringTok{"1"} \OtherTok{=} \StringTok{"blue"}\NormalTok{, }\StringTok{"0"} \OtherTok{=} \StringTok{"red"}\NormalTok{), }\AttributeTok{labels =} \FunctionTok{c}\NormalTok{(}\StringTok{"Regular"}\NormalTok{, }\StringTok{"Legendary"}\NormalTok{)) }\SpecialCharTok{+}
  \FunctionTok{theme\_bw}\NormalTok{()}

\NormalTok{sp\_attack }\OtherTok{\textless{}{-}} \FunctionTok{ggplot}\NormalTok{(}\FunctionTok{na.omit}\NormalTok{(df), }\FunctionTok{aes}\NormalTok{(}\AttributeTok{x =} \FunctionTok{factor}\NormalTok{(is\_legendary), }\AttributeTok{y =}\NormalTok{ sp\_attack, }\AttributeTok{fill =} \FunctionTok{factor}\NormalTok{(is\_legendary))) }\SpecialCharTok{+}
  \FunctionTok{geom\_boxplot}\NormalTok{() }\SpecialCharTok{+}
  \FunctionTok{labs}\NormalTok{(}\AttributeTok{title =} \StringTok{""}\NormalTok{, }\AttributeTok{x =} \StringTok{"Legendary Status"}\NormalTok{, }\AttributeTok{y =} \StringTok{"Special Attack"}\NormalTok{) }\SpecialCharTok{+}
  \FunctionTok{scale\_x\_discrete}\NormalTok{(}\AttributeTok{labels =} \FunctionTok{c}\NormalTok{(}\StringTok{"Regular"}\NormalTok{, }\StringTok{"Legendary"}\NormalTok{)) }\SpecialCharTok{+}
  \FunctionTok{scale\_fill\_manual}\NormalTok{(}\AttributeTok{values =} \FunctionTok{c}\NormalTok{(}\StringTok{"1"} \OtherTok{=} \StringTok{"blue"}\NormalTok{, }\StringTok{"0"} \OtherTok{=} \StringTok{"red"}\NormalTok{), }\AttributeTok{labels =} \FunctionTok{c}\NormalTok{(}\StringTok{"Regular"}\NormalTok{, }\StringTok{"Legendary"}\NormalTok{)) }\SpecialCharTok{+}
  \FunctionTok{theme\_bw}\NormalTok{()}

\NormalTok{sp\_defence }\OtherTok{\textless{}{-}} \FunctionTok{ggplot}\NormalTok{(}\FunctionTok{na.omit}\NormalTok{(df), }\FunctionTok{aes}\NormalTok{(}\AttributeTok{x =} \FunctionTok{factor}\NormalTok{(is\_legendary), }\AttributeTok{y =}\NormalTok{ sp\_defense, }\AttributeTok{fill =} \FunctionTok{factor}\NormalTok{(is\_legendary))) }\SpecialCharTok{+}
  \FunctionTok{geom\_boxplot}\NormalTok{() }\SpecialCharTok{+}
  \FunctionTok{labs}\NormalTok{(}\AttributeTok{title =} \StringTok{""}\NormalTok{, }\AttributeTok{x =} \StringTok{"Legendary Status"}\NormalTok{, }\AttributeTok{y =} \StringTok{"Special Defence"}\NormalTok{) }\SpecialCharTok{+}
  \FunctionTok{scale\_x\_discrete}\NormalTok{(}\AttributeTok{labels =} \FunctionTok{c}\NormalTok{(}\StringTok{"Regular"}\NormalTok{, }\StringTok{"Legendary"}\NormalTok{)) }\SpecialCharTok{+}
  \FunctionTok{scale\_fill\_manual}\NormalTok{(}\AttributeTok{values =} \FunctionTok{c}\NormalTok{(}\StringTok{"1"} \OtherTok{=} \StringTok{"blue"}\NormalTok{, }\StringTok{"0"} \OtherTok{=} \StringTok{"red"}\NormalTok{), }\AttributeTok{labels =} \FunctionTok{c}\NormalTok{(}\StringTok{"Regular"}\NormalTok{, }\StringTok{"Legendary"}\NormalTok{)) }\SpecialCharTok{+}
  \FunctionTok{theme\_bw}\NormalTok{()}

\NormalTok{speeed }\OtherTok{\textless{}{-}} \FunctionTok{ggplot}\NormalTok{(}\FunctionTok{na.omit}\NormalTok{(df), }\FunctionTok{aes}\NormalTok{(}\AttributeTok{x =} \FunctionTok{factor}\NormalTok{(is\_legendary), }\AttributeTok{y =}\NormalTok{ speed, }\AttributeTok{fill =} \FunctionTok{factor}\NormalTok{(is\_legendary))) }\SpecialCharTok{+}
  \FunctionTok{geom\_boxplot}\NormalTok{() }\SpecialCharTok{+}
  \FunctionTok{labs}\NormalTok{(}\AttributeTok{title =} \StringTok{""}\NormalTok{, }\AttributeTok{x =} \StringTok{"Legendary Status"}\NormalTok{, }\AttributeTok{y =} \StringTok{"Speed"}\NormalTok{) }\SpecialCharTok{+}
  \FunctionTok{scale\_x\_discrete}\NormalTok{(}\AttributeTok{labels =} \FunctionTok{c}\NormalTok{(}\StringTok{"Regular"}\NormalTok{, }\StringTok{"Legendary"}\NormalTok{)) }\SpecialCharTok{+}
  \FunctionTok{scale\_fill\_manual}\NormalTok{(}\AttributeTok{values =} \FunctionTok{c}\NormalTok{(}\StringTok{"1"} \OtherTok{=} \StringTok{"blue"}\NormalTok{, }\StringTok{"0"} \OtherTok{=} \StringTok{"red"}\NormalTok{), }\AttributeTok{labels =} \FunctionTok{c}\NormalTok{(}\StringTok{"Regular"}\NormalTok{, }\StringTok{"Legendary"}\NormalTok{)) }\SpecialCharTok{+}
  \FunctionTok{theme\_bw}\NormalTok{()}

\NormalTok{hpp }\OtherTok{\textless{}{-}} \FunctionTok{ggplot}\NormalTok{(}\FunctionTok{na.omit}\NormalTok{(df), }\FunctionTok{aes}\NormalTok{(}\AttributeTok{x =} \FunctionTok{factor}\NormalTok{(is\_legendary), }\AttributeTok{y =}\NormalTok{ hp, }\AttributeTok{fill =} \FunctionTok{factor}\NormalTok{(is\_legendary))) }\SpecialCharTok{+}
  \FunctionTok{geom\_boxplot}\NormalTok{() }\SpecialCharTok{+}
  \FunctionTok{labs}\NormalTok{(}\AttributeTok{title =} \StringTok{""}\NormalTok{, }\AttributeTok{x =} \StringTok{"Legendary Status"}\NormalTok{, }\AttributeTok{y =} \StringTok{"HP"}\NormalTok{) }\SpecialCharTok{+}
  \FunctionTok{scale\_x\_discrete}\NormalTok{(}\AttributeTok{labels =} \FunctionTok{c}\NormalTok{(}\StringTok{"Regular"}\NormalTok{, }\StringTok{"Legendary"}\NormalTok{)) }\SpecialCharTok{+}
  \FunctionTok{scale\_fill\_manual}\NormalTok{(}\AttributeTok{values =} \FunctionTok{c}\NormalTok{(}\StringTok{"1"} \OtherTok{=} \StringTok{"blue"}\NormalTok{, }\StringTok{"0"} \OtherTok{=} \StringTok{"red"}\NormalTok{), }\AttributeTok{labels =} \FunctionTok{c}\NormalTok{(}\StringTok{"Regular"}\NormalTok{, }\StringTok{"Legendary"}\NormalTok{)) }\SpecialCharTok{+}
  \FunctionTok{theme\_bw}\NormalTok{()}

\FunctionTok{grid.arrange}\NormalTok{(attack, defence, sp\_attack, sp\_defence, speeed, hpp, }\AttributeTok{ncol =} \DecValTok{2}\NormalTok{)}
\end{Highlighting}
\end{Shaded}

\includegraphics{pokemon_files/figure-latex/unnamed-chunk-13-1.pdf}
\#\#\# 2.6 Utilizing Machine Learning to Predict Which Pokémon are
Legendary or Not? \#\#\#\# 2.6.1 Splitting the Data into Training and
Testing

Training Data: 534 Pokémon

Testing Data: 267 Pokémon

\begin{Shaded}
\begin{Highlighting}[]
\NormalTok{training\_data }\OtherTok{\textless{}{-}} \FunctionTok{sample}\NormalTok{(}\DecValTok{1}\SpecialCharTok{:}\FunctionTok{nrow}\NormalTok{(df), }\DecValTok{2} \SpecialCharTok{*} \FunctionTok{nrow}\NormalTok{(df) }\SpecialCharTok{/} \DecValTok{3}\NormalTok{)}
\NormalTok{testing\_data }\OtherTok{\textless{}{-}} \FunctionTok{setdiff}\NormalTok{(}\DecValTok{1}\SpecialCharTok{:}\FunctionTok{nrow}\NormalTok{(df), training\_data)}

\NormalTok{legendary\_test }\OtherTok{\textless{}{-}}\NormalTok{ df}\SpecialCharTok{$}\NormalTok{is\_legendary[testing\_data]}
\end{Highlighting}
\end{Shaded}

\paragraph{2.6.2 Decision Tree Algorithm}\label{decision-tree-algorithm}

``Decision tree learning is a supervised learning approach used in
statistics, data mining and machine learning. In this formalism, a
classification or regression decision tree is used as a predictive model
to draw conclusions about a set of observations. Decision trees are
among the most popular machine learning algorithms given their
intelligibility and simplicity''
(\url{https://en.wikipedia.org/wiki/Decision_tree_learning}).

The decision tree for predicting whether Pokémon are legendary begins at
the root with the special attack attribute. It determines whether the
Pokémon's special attack stat is greater or less than 71.5. If it's less
than 71.5, the Pokémon is automatically deemed to not be legendary. If
it is, the tree branches off into different attributes including defense
and speed. Special attack is the most important metric, as it forms the
root of the tree, it's the first gate which determines whether a Pokémon
should be evaluated further. Defense and speed are also important
metrics for how this model determines if a Pokémon is legendary.

Ultimately, this systematic breakdown navigates through various metrics,
with each one acting as a gate leading to the Pokémon's classification.
Each decision point in the tree filters the Pokémon through key
criteria, with only those that meet all the thresholds being deemed
legendary.

Accuracy of Decision Tree Model: 0.895

\begin{Shaded}
\begin{Highlighting}[]
\FunctionTok{set.seed}\NormalTok{(}\DecValTok{200}\NormalTok{)}

\NormalTok{df}\SpecialCharTok{$}\NormalTok{is\_legendary }\OtherTok{\textless{}{-}} \FunctionTok{as.factor}\NormalTok{(df}\SpecialCharTok{$}\NormalTok{is\_legendary)}

\NormalTok{pokemon\_tree }\OtherTok{\textless{}{-}} \FunctionTok{tree}\NormalTok{(is\_legendary }\SpecialCharTok{\textasciitilde{}}\NormalTok{ ., }\AttributeTok{data =}\NormalTok{ df[training\_data, ], }\AttributeTok{na.action =}\NormalTok{ na.omit)}
\end{Highlighting}
\end{Shaded}

\begin{verbatim}
## Warning in tree(is_legendary ~ ., data = df[training_data, ], na.action =
## na.omit): NAs introduced by coercion
\end{verbatim}

\begin{Shaded}
\begin{Highlighting}[]
\NormalTok{colors }\OtherTok{\textless{}{-}} \FunctionTok{brewer.pal}\NormalTok{(}\DecValTok{3}\NormalTok{, }\StringTok{"Set3"}\NormalTok{)}
\FunctionTok{plot}\NormalTok{(pokemon\_tree, }\AttributeTok{col =}\NormalTok{ colors, }\AttributeTok{lwd =} \DecValTok{2}\NormalTok{, }\AttributeTok{main =} \StringTok{"Decision Tree for Legendary Pokémon Prediction"}\NormalTok{)}
\FunctionTok{text}\NormalTok{(pokemon\_tree, }\AttributeTok{pretty =} \DecValTok{0}\NormalTok{, }\AttributeTok{col =} \StringTok{"blue"}\NormalTok{, }\AttributeTok{cex =} \FloatTok{0.45}\NormalTok{)}
\FunctionTok{grid}\NormalTok{()}
\end{Highlighting}
\end{Shaded}

\includegraphics{pokemon_files/figure-latex/unnamed-chunk-15-1.pdf}

\begin{Shaded}
\begin{Highlighting}[]
\NormalTok{predictions }\OtherTok{\textless{}{-}} \FunctionTok{predict}\NormalTok{(pokemon\_tree, }\AttributeTok{newdata =}\NormalTok{ df[testing\_data, ], }\AttributeTok{type =} \StringTok{"class"}\NormalTok{)}
\end{Highlighting}
\end{Shaded}

\begin{verbatim}
## Warning in pred1.tree(object, tree.matrix(newdata)): NAs introduced by coercion
\end{verbatim}

\begin{Shaded}
\begin{Highlighting}[]
\NormalTok{confusion\_matrix }\OtherTok{\textless{}{-}} \FunctionTok{table}\NormalTok{(predictions, legendary\_test)}

\NormalTok{accuracy }\OtherTok{\textless{}{-}} \FunctionTok{sum}\NormalTok{(}\FunctionTok{diag}\NormalTok{(confusion\_matrix)) }\SpecialCharTok{/} \FunctionTok{sum}\NormalTok{(confusion\_matrix)}

\FunctionTok{cat}\NormalTok{(}\StringTok{"Accuracy:"}\NormalTok{, accuracy, }\StringTok{"}\SpecialCharTok{\textbackslash{}n}\StringTok{"}\NormalTok{)}
\end{Highlighting}
\end{Shaded}

\begin{verbatim}
## Accuracy: 0.9101124
\end{verbatim}

\paragraph{2.6.3 Random Forest Algorithm}\label{random-forest-algorithm}

``Random forests or random decision forests is an ensemble learning
method for classification, regression and other tasks that works by
creating a multitude of decision trees during training. For
classification tasks, the output of the random forest is the class
selected by most trees. For regression tasks, the output is the average
of the predictions of the trees
(\url{https://en.wikipedia.org/wiki/Random_forest}).''

\begin{Shaded}
\begin{Highlighting}[]
\NormalTok{df}\SpecialCharTok{$}\NormalTok{is\_legendary }\OtherTok{\textless{}{-}} \FunctionTok{as.factor}\NormalTok{(df}\SpecialCharTok{$}\NormalTok{is\_legendary)}

\NormalTok{pokemon\_rf }\OtherTok{\textless{}{-}} \FunctionTok{randomForest}\NormalTok{(is\_legendary }\SpecialCharTok{\textasciitilde{}}\NormalTok{ ., }\AttributeTok{data =}\NormalTok{ df[training\_data, ], }\AttributeTok{importance =} \ConstantTok{TRUE}\NormalTok{, }\AttributeTok{na.action =}\NormalTok{ na.omit, }\AttributeTok{type =} \StringTok{"classification"}\NormalTok{)}

\FunctionTok{print}\NormalTok{(pokemon\_rf)}
\end{Highlighting}
\end{Shaded}

\begin{verbatim}
## 
## Call:
##  randomForest(formula = is_legendary ~ ., data = df[training_data,      ], importance = TRUE, type = "classification", na.action = na.omit) 
##                Type of random forest: classification
##                      Number of trees: 500
## No. of variables tried at each split: 3
## 
##         OOB estimate of  error rate: 5.95%
## Confusion matrix:
##     0  1 class.error
## 0 474  2 0.004201681
## 1  29 16 0.644444444
\end{verbatim}

The confusion matrix indicates that:

\begin{itemize}
\item
  477 instances were correctly classified as class 0.
\item
  4 instances were incorrectly classified as class 0 (false negatives).
\item
  29 instances were incorrectly classified as class 1 (false positives).
\item
  12 instances were correctly classified as class 1.
\end{itemize}

The class error rates for each class are also provided:

\begin{itemize}
\item
  Class 0: 0.00845666 (approximately 0.83\%)
\item
  Class 1: 0.65909091 (approximately 68.29\%)
\end{itemize}

Accuracy = (True Positives + True Negatives) / (Total Instances)

Accuracy = (12 + 477) / 522 ≈ 0.9377

Accuracy of Random Forest Model: 0.937

\subparagraph{2.6.3.1 Most Important
Metrics}\label{most-important-metrics}

\begin{itemize}
\item
  Mean Decrease Accuracy: a metric that measures how much a model's
  accuracy decreases when a variable is removed.
\item
  Mean Decrease Gini: measures how important a variable in a random
  forest model, by quantifying how much it contributes to the
  homogeneity of the model's nodes.
\end{itemize}

As shown by the output below, the most crucial three metrics used by the
Random Forest model to successfully predict whether a Pokémon are
special attack, weight and speed.

\begin{Shaded}
\begin{Highlighting}[]
\FunctionTok{varImpPlot}\NormalTok{(pokemon\_rf)}
\end{Highlighting}
\end{Shaded}

\includegraphics{pokemon_files/figure-latex/unnamed-chunk-18-1.pdf} \#
3. Conclusion

\begin{Shaded}
\begin{Highlighting}[]
\NormalTok{knitr}\SpecialCharTok{::}\FunctionTok{include\_graphics}\NormalTok{(}\StringTok{"legendary{-}pokemon{-}55326671.jpg"}\NormalTok{)}
\end{Highlighting}
\end{Shaded}

\includegraphics{legendary-pokemon-55326671.jpg}

In conclusion, as shown by the ROC Curve graph below, the Random Forest
model (blue) is clearly more accurate than the Decision Tree model
(red). The closer the curve is to the top left corner, the more accurate
and efficient the model is. The Random Forest model achieved higher
sensitivity and specificty compared to the other one, which thus
indicates that it's more efficient at correctly classifying both
legendary and regular Pokémon.

The higher accuracy, is due to the nature of the Random Forest
algorithm, which combined multiple trees which produced more reliable
predictions; the Decision Tree algorithm is much more simplistic and
lacks the robustness as Random Forest. Through comparing two different
models, it highlighted the importance of model selection and the impact
of algorithm complexity on predictive performance.

The most crucial metrics across both models in determining Legendary
Pokémon are the following:

\begin{itemize}
\item
  Special Attack
\item
  Weight (KG)
\item
  Speed
\end{itemize}

Overall this simplistic project explored and introduced the fundamental
concepts of data analysis, visualization and machine learning with R. It
acted as a simplistic gateway offering practical, hands-on experience in
applying these techniques. The skills and insights gained here lay a
solid foundation for more advanced explorations in the exciting fields
of data analysis and data science.

\begin{Shaded}
\begin{Highlighting}[]
\CommentTok{\# Predict probabilities for both models}
\NormalTok{rf\_prob }\OtherTok{\textless{}{-}} \FunctionTok{predict}\NormalTok{(pokemon\_rf, df[testing\_data, ], }\AttributeTok{type =} \StringTok{"prob"}\NormalTok{)[, }\DecValTok{2}\NormalTok{]}
\NormalTok{tree\_prob }\OtherTok{\textless{}{-}} \FunctionTok{predict}\NormalTok{(pokemon\_tree, df[testing\_data, ], }\AttributeTok{type =} \StringTok{"vector"}\NormalTok{)[, }\DecValTok{2}\NormalTok{]}
\end{Highlighting}
\end{Shaded}

\begin{verbatim}
## Warning in pred1.tree(object, tree.matrix(newdata)): NAs introduced by coercion
\end{verbatim}

\begin{Shaded}
\begin{Highlighting}[]
\CommentTok{\# Generate ROC curves}
\NormalTok{rf\_roc }\OtherTok{\textless{}{-}} \FunctionTok{roc}\NormalTok{(df}\SpecialCharTok{$}\NormalTok{is\_legendary[testing\_data], rf\_prob)}
\end{Highlighting}
\end{Shaded}

\begin{verbatim}
## Setting levels: control = 0, case = 1
\end{verbatim}

\begin{verbatim}
## Setting direction: controls < cases
\end{verbatim}

\begin{Shaded}
\begin{Highlighting}[]
\NormalTok{tree\_roc }\OtherTok{\textless{}{-}} \FunctionTok{roc}\NormalTok{(df}\SpecialCharTok{$}\NormalTok{is\_legendary[testing\_data], tree\_prob)}
\end{Highlighting}
\end{Shaded}

\begin{verbatim}
## Setting levels: control = 0, case = 1
## Setting direction: controls < cases
\end{verbatim}

\begin{Shaded}
\begin{Highlighting}[]
\CommentTok{\# Plot ROC curves}
\FunctionTok{plot}\NormalTok{(rf\_roc, }\AttributeTok{col =} \StringTok{"blue"}\NormalTok{, }\AttributeTok{main =} \StringTok{"ROC Curves for Random Forest and Decision Tree"}\NormalTok{)}
\FunctionTok{lines}\NormalTok{(tree\_roc, }\AttributeTok{col =} \StringTok{"red"}\NormalTok{)}
\FunctionTok{legend}\NormalTok{(}\StringTok{"bottomright"}\NormalTok{, }\AttributeTok{legend =} \FunctionTok{c}\NormalTok{(}\StringTok{"Random Forest"}\NormalTok{, }\StringTok{"Decision Tree"}\NormalTok{), }\AttributeTok{col =} \FunctionTok{c}\NormalTok{(}\StringTok{"blue"}\NormalTok{, }\StringTok{"red"}\NormalTok{), }\AttributeTok{lwd =} \DecValTok{2}\NormalTok{)}
\end{Highlighting}
\end{Shaded}

\includegraphics{pokemon_files/figure-latex/unnamed-chunk-20-1.pdf}

\end{document}
